% Created 2026-02-13 Fri 13:58
% Intended LaTeX compiler: pdflatex
\documentclass[11pt,a4paper]{article}
\usepackage[utf8]{inputenc}
\usepackage[T1]{fontenc}
\usepackage{graphicx}
\usepackage{longtable}
\usepackage{wrapfig}
\usepackage{rotating}
\usepackage[normalem]{ulem}
\usepackage{amsmath}
\usepackage{amssymb}
\usepackage{capt-of}
\usepackage{hyperref}
\usepackage{amsmath,amssymb}
\usepackage{graphicx}
\usepackage[margin=2.5cm]{geometry}
\usepackage{booktabs}
\usepackage{natbib}
\bibliographystyle{abbrvnat}
\newcommand{\fadox}{[\mathrm{FAD}^{\cdot-}\;\mathrm{O}_2^{\cdot-}]}
\newcommand{\fadtrp}{[\mathrm{FAD}^{\cdot-}\;\mathrm{TrpH}^{\cdot+}]}
\usepackage{xcolor}
\DeclareMathOperator{\Tr}{Tr}
\DeclareMathOperator{\argop}{arg}
\newcommand{\eg}{e.g.}
\newcommand{\ie}{i.e.}
\newcommand{\Ncry}{N_{\mathrm{cry}}}
\newcommand{\PhiS}{\Phi_S}
\newcommand{\thetahat}{\hat{\theta}}
\author{mu2tau}
\date{2026-02-13}
\title{From quantum compass to neural path integrator: a three-level model of radical-pair magnetoreception in navigating insects}
\hypersetup{
 pdfauthor={mu2tau},
 pdftitle={From quantum compass to neural path integrator: a three-level model of radical-pair magnetoreception in navigating insects},
 pdfkeywords={},
 pdfsubject={},
 pdfcreator={Emacs 30.2 (Org mode 9.7.11)}, 
 pdflang={English}}
\begin{document}

\maketitle
\setcounter{tocdepth}{2}
\tableofcontents

\begin{abstract}
The radical-pair mechanism (RPM) is the leading hypothesis for the magnetic
compass of migratory animals, yet no published model links quantum spin
chemistry to neural path integration in a behaving agent. We present a
three-level computational model that bridges this gap: (i) a Haberkorn
master equation with Lindblad relaxation generates the singlet-yield
compass signal from the spin dynamics of flavin--superoxide and
flavin--tryptophan radical pairs in cryptochrome; (ii) an eight-neuron
ring attractor converts the noisy compass signal into a stable heading
estimate; (iii) a CPU4 leaky integrator accumulates heading-weighted
displacement for path integration, and a persistent random walker uses
both compass and home-vector steering. Systematic parameter sweeps across
seven compass models reveal a sharp navigation threshold at contrast
\(C \sim 0.1\): above this value all models perform identically and motor
noise dominates; below it, compass noise is the binding constraint.
Three suppression mechanisms (spin relaxation, rate asymmetry,
orientational disorder) reduce the theoretical contrast by \(\sim 50\%\),
leaving \(\fadox\) with a 3\(\times\) safety margin but
\(\fadtrp\) marginal at the threshold. We report a novel
analytical result: constant compass bias cancels exactly in the CPU4
architecture due to same-frame integration and readout, regardless of
path tortuosity. Spatially varying anomalies from geological sources
break this cancellation but remain second-order within typical foraging
radii. The binding constraints on path integration are memory leak
(\(\lambda T < 1\)) and compass noise, not field inhomogeneity.
\end{abstract}
\section{Introduction}
\label{sec:org47e8b9e}
The hypothesis that animals sense Earth's magnetic field through
quantum spin dynamics in radical pairs dates to Schulten, Swenberg and
Weller \cite{schulten1978}, who proposed that the interconversion of
singlet and triplet electronic spin states in photochemically generated
radical pairs is sensitive to weak magnetic fields. Ritz, Adem and
Schulten \cite{ritz2000} identified the blue-light photoreceptor
cryptochrome as the candidate magnetoreceptor, launching two decades
of experimental and theoretical investigation reviewed by Hore and
Mouritsen \cite{hore2016}.

The central physics is well established. A radical pair born in a
spin-correlated singlet state undergoes coherent singlet--triplet
interconversion driven by anisotropic hyperfine coupling to nearby
nuclear spins. The external magnetic field defines a quantisation axis
that modulates the mixing rate as a function of the angle between the
field and the molecular frame, producing an angular dependence in the
singlet yield --- the compass signal. At Earth-field strength
(\(\sim 50\,\mu\mathrm{T}\)), the electron Zeeman splitting
(\(\sim 1.4\,\mathrm{MHz}\)) is comparable to the dominant hyperfine
couplings in flavin adenine dinucleotide (FAD), enabling
field-dependent modulation provided the radical pair persists for at
least one Larmor period (\(\gtrsim 1\,\mu\mathrm{s}\))
\cite{hore2016}.

Two radical-pair candidates have received sustained attention. The
flavin--tryptophan pair \(\fadtrp\), formed through the
conserved tryptophan tetrad in cryptochrome, produces a compass
anisotropy \(\Delta\Phi_S/\bar{\Phi}_S \sim 0.1\text{--}1\%\)
\cite{worster2017}. The alternative flavin--superoxide pair
\(\fadox\), proposed by Lee et al. \cite{lee2014},
yields anisotropy two orders of magnitude larger because only the FAD
radical carries hyperfine-active nuclei. The experimental picture
remains contested: Bradlaugh et al. \cite{bradlaugh2023} demonstrated
cryptochrome-dependent magnetoreception in \emph{Drosophila}, while
Bassetto et al. \cite{bassetto2023} found no evidence in 97,658 flies
under similar conditions.

On the neural side, the insect central complex (CX) is the established
navigation centre, housing a ring attractor for heading representation
\cite{kim2017,seelig2015} and CPU4/CPU1 neurons for path integration
\cite{stone2017}. However, \emph{no magnetically responsive neuron has been
identified within the CX} \cite{nordmann2017}, and no published model
couples a radical-pair compass to a path-integrating agent navigating
a landscape.

This paper fills that gap. We construct a three-level model ---
quantum spin dynamics, ring attractor compass, path-integrating
navigation agent --- and systematically characterise the conditions
under which the compass supports effective navigation. The analysis
proceeds in three directions: (A) robustness to geological magnetic
anomalies, (B) path integration via the CPU4 circuit, and (\(A \!\times\! B\))
their interaction. We identify the compass contrast threshold (\(C \sim
0.1\)) as the critical gatekeeper, report an exact cancellation of
constant compass bias in path integration, and establish memory leak
rather than field inhomogeneity as the binding constraint on homing
accuracy.
\section{Model}
\label{sec:orgb4f2697}
\subsection{Radical-pair spin dynamics}
\label{sec:org57901cc}
We consider two unpaired electrons (radicals A and B), each coupled to
local nuclear spins via hyperfine interaction, in an external field
\(\mathbf{B}_0\). The spin Hamiltonian is

\begin{equation}
\label{eq:hamiltonian}
\hat{H} = -\gamma_e \mathbf{B}_0 \cdot (\hat{\mathbf{S}}_A + \hat{\mathbf{S}}_B)
         + \sum_{i \in A} \hat{\mathbf{S}}_A \cdot \mathbf{A}_i \cdot \hat{\mathbf{I}}_i
         + \sum_{j \in B} \hat{\mathbf{S}}_B \cdot \mathbf{A}_j \cdot \hat{\mathbf{I}}_j
         + J\left(\tfrac{1}{4} + \hat{\mathbf{S}}_A \cdot \hat{\mathbf{S}}_B\right)
         + \hat{H}_{dd}
\end{equation}

where \(\gamma_e = g_e \mu_B / \hbar \approx 1.761 \times 10^{11}\)
rad s\(^{-1}\) T\(^{-1}\), \(\mathbf{A}_i\) are hyperfine coupling tensors,
\(J\) is the exchange coupling, and \(\hat{H}_{dd}\) is the
electron--electron dipolar coupling. At \(50\,\mu\mathrm{T}\) the
dominant FAD hyperfine couplings are N5 (\(a_{\mathrm{iso}} \approx
523\,\mu\mathrm{T}\), \(\sim 14.7\,\mathrm{MHz}\)) and N10
(\(a_{\mathrm{iso}} \approx 189\,\mu\mathrm{T}\), \(\sim
5.3\,\mathrm{MHz}\)).

The density matrix evolves under the Haberkorn master equation with
Lindblad relaxation:

\begin{equation}
\label{eq:master}
\frac{d\hat{\rho}}{dt} = -\frac{i}{\hbar}[\hat{H}, \hat{\rho}]
  - \frac{k_S}{2}\{\hat{P}_S, \hat{\rho}\}
  - \frac{k_T}{2}\{\hat{P}_T, \hat{\rho}\}
  - \hat{\mathcal{R}}[\hat{\rho}]
\end{equation}

where \(\hat{P}_S = \lvert S\rangle\langle S\rvert\) and \(\hat{P}_T = \sum_m \lvert T_m\rangle\langle T_m\rvert\) are the singlet and triplet projectors, \(k_S\)
and \(k_T\) are spin-selective recombination rates, and
\(\hat{\mathcal{R}}\) is a Lindblad relaxation superoperator with
\(T_1 = T_2\) parameterising the dephasing timescale. The singlet yield
--- the compass observable --- is

\begin{equation}
\label{eq:singlet-yield}
\PhiS(\theta) = k_S \int_0^{\infty} \Tr[\hat{P}_S \hat{\rho}(t)]\, dt
\end{equation}

where \(\theta\) is the angle between \(\mathbf{B}_0\) and the molecular
frame. We define the compass contrast as

\begin{equation}
\label{eq:contrast}
C = \frac{\PhiS^{\max} - \PhiS^{\min}}{\PhiS^{\max} + \PhiS^{\min}}
\end{equation}

We implement seven compass models spanning three Hilbert-space
dimensions: a toy model (2 electrons + 1 nucleus, dim = 8), an
intermediate model (2 electrons + 2 nuclei, dim = 16), and their
variants for both \(\fadox\) (no nuclei on radical B)
and \(\fadtrp\) (additional \$\(\beta\)\$-proton hyperfine
couplings). The singlet yield is well approximated by
\(\Phi_S(\theta) \approx \bar{\Phi}_S + \delta\Phi_S \cos^2\theta\), with
\(L = 0, 2\) Legendre components capturing 99.9\% of the anisotropy.
\subsection{Neural compass: ring attractor}
\label{sec:org4256fba}
The compass signal from the radical-pair layer is transduced into a
heading estimate by an 8-neuron ring attractor modelled on the
ellipsoid body (EB) of the insect central complex. Each neuron \(i\) has
preferred heading \(\theta_i = 2\pi i/N\) and firing rate \(r_i(t)\)
evolving as

\begin{equation}
\label{eq:ring-attractor}
\tau_r \frac{dr_i}{dt} = -r_i + \left[\sum_j W_{ij} r_j + I_i^{\mathrm{mag}}(t) - b\right]^+
\end{equation}

where \(W_{ij} = w_{\mathrm{exc}} \cos(\theta_i - \theta_j) -
w_{\mathrm{inh}}\) combines local excitation with global inhibition
(the \(\Delta 7\) neuron equivalent), \(I_i^{\mathrm{mag}}\) is the
magnetic input from the cryptochrome array mapped onto the ring,
\(b\) is a threshold, and \([\cdot]^+ = \max(0, \cdot)\). The heading
estimate is decoded by population vector:
\(\thetahat = \argop(\sum_i r_i e^{i\theta_i})\).

The compass noise propagated to the heading estimate scales as

\begin{equation}
\label{eq:compass-noise}
\sigma_{\mathrm{compass}} \approx \frac{1}{C\sqrt{\Ncry}}
\end{equation}

where \(C\) is the contrast (Eq. \ref{eq:contrast}) and \(\Ncry\) is the
number of cryptochrome molecules contributing to the population
average. At \(C = 0.15\) with \(\Ncry = 50\): \(\sigma_{\mathrm{compass}}
\approx 0.9\,\mathrm{rad} \approx 53^{\circ}\); at \(C = 0.01\):
\(\sigma_{\mathrm{compass}} \approx 14\,\mathrm{rad}\), well beyond
the ring attractor's tracking ability.
\subsection{Navigation: persistent random walk}
\label{sec:org16ac780}
The agent navigates a 2D landscape as a biased persistent random
walker. Position \((x, y)\) and heading \(\theta\) evolve as

\begin{equation}
\label{eq:heading-update}
\theta_{t+dt} = \theta_t + \kappa \sin(\theta_{\mathrm{goal}} - \thetahat_t)\,dt + \sigma_\theta \sqrt{dt}\, \eta_\theta
\end{equation}

where \(\kappa\) is the steering gain, \(\theta_{\mathrm{goal}}\) the
desired magnetic heading, \(\thetahat_t\) the ring-attractor estimate,
and \(\sigma_\theta\) combines compass and motor noise. The spatial
updates follow \(dx = v\cos\theta\,dt\), \(dy = v\sin\theta\,dt\) with
additive translational noise.

The navigation efficiency is characterised by the Peclet number

\begin{equation}
\label{eq:peclet}
Pe = \frac{\kappa L}{\sigma_\theta^2}
\end{equation}

where \(L\) is the distance to goal. Navigation transitions from
effective to diffusive at \$Pe \(\sim\) 100\$--\(1000\) in ensembles of 200
bugs per parameter point.
\subsection{Path integration: CPU4 circuit}
\label{sec:org1862aca}
We implement the CPU4 path integration circuit of Stone et al.
\cite{stone2017}. Eight CPU4 neurons with preferred directions
\(\phi_i = 2\pi i/8\) accumulate heading-weighted displacement:

\begin{equation}
\label{eq:cpu4}
M_i(t + dt) = (1 - \lambda\,dt)\,M_i(t) + v\,[\cos(\thetahat(t) - \phi_i)]_+\,dt
\end{equation}

where \(\thetahat\) is the compass-estimated heading (noisy, potentially
biased), \([\cdot]_+\) is half-wave rectification, \(v\) is speed, and
\(\lambda\) is the memory leak rate. The home vector is decoded by
population vector:

\begin{equation}
\label{eq:home-vector}
\theta_{\mathrm{home}} = \argop\!\left(-\sum_i M_i\, e^{i\phi_i}\right)
\end{equation}

During the homing phase, the steering law (Eq. \ref{eq:heading-update})
is applied with \(\theta_{\mathrm{goal}}\) replaced by
\(\theta_{\mathrm{home}}\).
\section{Results}
\label{sec:org6847fbd}
\subsection{Compass contrast and the navigation threshold}
\label{sec:orgecec3d3}
Sweeping compass contrast \(C\) across all seven models with \(\Ncry = 50\)
reveals a sharp two-regime behaviour (Figure \ref{fig:orgc75f45f}).
Above \(C \gtrsim 0.1\), all models produce indistinguishable navigation
performance: mean heading error \(\lesssim 10^{\circ}\), and the compass is not
the bottleneck --- motor noise dominates. Below \(C \lesssim 0.1\),
performance degrades rapidly and every factor of 2 in contrast matters.

This reduces the \(\fadtrp\) versus
\(\fadox\) debate to a single question: \emph{does the
compass contrast in vivo exceed \(C \sim 0.1\)?} The toy \(\fadox\) model gives \(C_0 = 0.39\), safely above; the toy
\(\fadtrp\) gives \(C_0 = 0.21\), above but vulnerable
to suppression.

\begin{figure}[htbp]
\centering
\includegraphics[width=0.8\textwidth]{fig_differentiate.png}
\caption{\label{fig:orgc75f45f}Model differentiation at the navigation phase boundary. Above \(C \sim 0.1\) (dashed line) all seven compass models yield identical heading errors; below it, higher-contrast models progressively outperform lower-contrast ones.}
\end{figure}
\subsection{Robustness budget: the safety margin}
\label{sec:orgc94ae69}
Three physically motivated suppression mechanisms reduce the
theoretical compass contrast toward the threshold:

\textbf{Spin relaxation.} Lindblad dephasing with \(T_1 = T_2\) suppresses the
contrast progressively. At \(T_2 = 3\,\mu\mathrm{s}\): toy
\(\fadox\) drops from \(C_0 = 0.39\) to \(C = 0.28\) (a
factor 0.68); intermediate \(\fadtrp\) from 0.25 to
0.16 (factor 0.63). Models with more nuclei are \emph{more} vulnerable
because larger Hilbert spaces have more coherences exposed to
dephasing.

\textbf{Rate asymmetry.} Unequal singlet/triplet recombination rates
(\(k_T \neq k_S\)) affect the absolute anisotropy \(\delta =
\Phi_S^{\max} - \Phi_S^{\min}\) rather than the relative contrast.
The anisotropy peaks at \(k_T/k_S \approx 0.5\) and is flat across
0.3--1.5. The equal-rate assumption is near-optimal.

\textbf{Orientational disorder.} Since the \(L = 0, 2\) Legendre components
capture 99.9\% of the anisotropy, molecular orientational averaging
multiplies the contrast by \(\langle P_2(\cos\Delta\theta)\rangle\).
At \(\sigma_{\mathrm{orient}} = 20^{\circ}\): \(\langle P_2 \rangle = 0.68\)
(32\% suppression).

The cumulative suppression is \(\sim 50\%\) for both radical pairs.
Starting from the intermediate model: \(\fadox\)
reaches \(C \approx 0.22\) (2.2\(\times\) above threshold);
\(\fadtrp\) reaches \(C \approx 0.10\) (right at
threshold). Figure \ref{fig:org2a34558} shows the safety margin as a
function of \(T_2\). \(\fadox\) maintains \(\geq 2.7\times\)
margin at \(T_2 = 3\,\mu\mathrm{s}\) and does not cross the
threshold until \(T_2 \leq 0.55\,\mu\mathrm{s}\).
\(\fadtrp\) crosses at \(T_2 \approx
1\,\mu\mathrm{s}\).

\begin{figure}[htbp]
\centering
\includegraphics[width=0.8\textwidth]{fig_relax_margin.png}
\caption{\label{fig:org2a34558}Safety margin versus spin relaxation time \(T_2\). The navigation threshold \(C = 0.1\) is shown as a horizontal dashed line. \(\fadox\) models (blue/green) maintain a \(\geq 3\times\) margin; \(\fadtrp\) models (orange/red) are marginal.}
\end{figure}
\subsection{Heading-following through anomaly fields}
\label{sec:org8ef6ce7}
Real landscapes harbour magnetic anomalies from geological sources.
We model these as buried vertical magnetic dipoles (volcanic
intrusions), linear faults, and regional gradients. The field
deviation \(\delta\varphi(x,y)\) is pre-computed on a \(150 \times 150\)
grid and bilinearly interpolated during simulation.

The compass is remarkably robust (Figure \ref{fig:orgd572c03}).
Baseline heading error is \(7.2^{\circ}\) in a uniform field. Adding 20 dipoles
of \(12\,\mu\mathrm{T}\) peak (50\% of the horizontal background)
raises this to only \(12.1^{\circ}\) --- well below the \(30^{\circ}\) navigation
breakdown threshold. Even extreme anomaly fields (\(\sim 50\%\) of
\(B_h\)) add at most \(\sim 5^{\circ}\) to the mean heading error. Three
mechanisms explain this robustness:

\begin{enumerate}
\item \textbf{Quasi-static tracking.} The steering time constant (\(1/\kappa
   \approx 0.5\,\mathrm{s}\)) is much shorter than the anomaly
crossing time (\$\(\sim\) d/v \(\approx\) 80\$--\(160\,\mathrm{s}\)). The bug
continuously re-aligns to the local field.
\item \textbf{Random cancellation.} Dipoles with random-sign moments create
oscillating perturbations that partially cancel over the path.
\item \textbf{Heading versus path.} The compass measures heading relative to the
local field. Anomalies deflect the geographic path but preserve
magnetic heading accuracy.
\end{enumerate}

\begin{figure}[htbp]
\centering
\includegraphics[width=0.85\textwidth]{fig_anomaly_sweep.png}
\caption{\label{fig:orgd572c03}Navigation error versus anomaly strength and density. Left: increasing dipole strength at fixed density. Right: increasing density at fixed strength. The \(30^{\circ}\) breakdown threshold (dashed) is never approached. Compass model differences vanish in the anomaly response.}
\end{figure}
\subsection{Path integration enables homing}
\label{sec:org048a245}
The CPU4 path integrator (Eq. \ref{eq:cpu4}) coupled to the
radical-pair compass enables homing from exploration. With a perfect
integrator (\(\lambda = 0\)), even 2000 s of random foraging produces
only \(\sim 4\) body-lengths (BL) homing error (Table \ref{tab:orge56d7eb}).
Heading reversal, the memoryless alternative, gives 2.5\(\times\) worse
performance at high compass noise (\(\sigma_\theta = 1.0\)).

\begin{table}[htbp]
\caption{\label{tab:orge56d7eb}Homing error (body-lengths) by strategy and compass noise level (\(\lambda = 0\), \(\Ncry = 50\), \(C = 0.15\)).}
\centering
\begin{tabular}{llll}
\toprule
Strategy & \(\sigma_\theta = 0.3\) & \(\sigma_\theta = 1.0\) & \(\sigma_\theta = 2.0\)\\
\midrule
PI (straight) & 3.2  BL & 3.3  BL & 5.3  BL\\
Reversal (straight) & 4.3  BL & 8.3  BL & 13.2  BL\\
PI after 2000 s explore & 4.1  BL & 3.9  BL & 4.0  BL\\
\bottomrule
\end{tabular}
\end{table}

The deepest result concerns compass bias. A constant bias \(b\) rotates
the heading estimate uniformly: \(\thetahat = \theta + b\). The CPU4
accumulates displacement in this biased compass frame, and the homing
readout also operates in the same frame. The bias therefore cancels
exactly:

\begin{equation}
\label{eq:same-frame}
\theta_{\mathrm{err}} = \theta_{\mathrm{home}}^{\mathrm{biased}} - \thetahat^{\mathrm{biased}} = (\theta_{\mathrm{home}} + b) - (\theta + b) = \theta_{\mathrm{home}} - \theta
\end{equation}

This \emph{same-frame cancellation} holds regardless of path tortuosity,
bias magnitude (tested up to \(45^{\circ}\)), or compass model. It is a
symmetry of the CPU4 architecture, not a feature of any particular
parameter regime. Figure \ref{fig:orgb5a7d55} (left panel) confirms
the null result: homing error is flat at \(\sim 2.4\) BL across
\$0\textsuperscript{\^{}}\$--\(45^{\circ}\) bias.

\begin{figure}[htbp]
\centering
\includegraphics[width=0.85\textwidth]{fig_pi_bias_leak.png}
\caption{\label{fig:orgb5a7d55}Left: constant compass bias has zero effect on PI homing accuracy (same-frame cancellation). Right: CPU4 memory leak \(\lambda\) progressively degrades homing; the transition occurs at \(\lambda T \sim 1\).}
\end{figure}
\subsection{Memory leak is the binding constraint}
\label{sec:org7b8c782}
The critical dimensionless number for path integration is \(\lambda T\),
the product of the CPU4 leak rate and the foraging duration. A phase
diagram (Figure \ref{fig:orgf3fc2ab}) reveals two regimes:

\begin{itemize}
\item \(\lambda T \ll 1\) (bottom-left): PI memory persists, homing error
\(\lesssim 5\) BL.
\item \(\lambda T \gg 1\) (top-right): memory has decayed, homing error
\(\gtrsim 50\) BL --- the bug is effectively amnesic.
\end{itemize}

The critical leak rate scales as

\begin{equation}
\label{eq:critical-leak}
\lambda^* \approx \frac{1}{T_{\mathrm{explore}}}
\end{equation}

For \(T_{\mathrm{explore}} = 500\,\mathrm{s}\):
\(\lambda^* \approx 0.002\,\mathrm{s}^{-1}\). Even modest leak
(\(\lambda = 0.001\,\mathrm{s}^{-1}\)) degrades homing from 2.4 BL to
22.2 BL at 500 s foraging. Memory leak, not compass quality, is the
binding constraint once the contrast exceeds the \(C \sim 0.1\)
threshold.

\begin{figure}[htbp]
\centering
\includegraphics[width=0.7\textwidth]{fig_pi_phase.png}
\caption{\label{fig:orgf3fc2ab}Phase diagram: CPU4 leak rate \(\lambda\) versus exploration time \(T_{\mathrm{explore}}\). Colour encodes mean homing error (BL). The \(\lambda T = 1\) contour (dashed) cleanly separates the ``PI works'' regime (below) from the ``amnesic'' regime (above).}
\end{figure}
\subsection{Double robustness of path integration to anomalies}
\label{sec:org7c362d6}
Spatially varying magnetic anomalies break the same-frame cancellation
(Eq. \ref{eq:same-frame}) because the bug encounters different
\(\delta\varphi(x,y)\) during outbound exploration and return homing.
The compass heading estimate in an anomaly field becomes
\(\thetahat = \theta + \eta_{\mathrm{compass}} + b +
\delta\varphi(x,y)\), and the spatially varying component does not
cancel between integration and readout.

Nevertheless, the effect is second-order (Figure
\ref{fig:org3fe58eb}). With 10 dipoles of \(12\,\mu\mathrm{T}\)
and \(T_{\mathrm{explore}} = 500\,\mathrm{s}\), the PI homing error
increases from 2.4 BL to only 2.6 BL at \(\sigma_\theta = 0.5\).
Three reinforcing mechanisms explain this:

\begin{enumerate}
\item \textbf{Short exploration radius.} With \(\sigma_\theta = 0.5\), the bug
explores \$\(\sim\) 30\$--50 BL from home. With 10 dipoles in a
\(1000 \times 1000\) BL landscape (mean spacing \(\sim 300\) BL), most
bugs never encounter a dipole.
\item \textbf{Local same-frame cancellation.} Within the exploration radius, the
spatially varying component is small; the locally uniform component
(which cancels) dominates.
\item \textbf{Noise masking.} For low-contrast compasses (\(C = 0.01\),
\(\sigma_{\mathrm{compass}} \approx 65^{\circ}\)), anomaly deviations
(\(\lesssim 13^{\circ}\)) are buried in compass noise.
\end{enumerate}

Anomaly density matters more than strength: 5--10 dipoles of
\(12\,\mu\mathrm{T}\) are negligible (\(\sim 2.6\) BL), but 50
dipoles raise the error to 17.1 BL. A counterintuitive finding:
persistent walkers (\(\sigma_\theta = 0.3\)) are \emph{more} vulnerable than
random walkers (\(\sigma_\theta = 1.0\)), because they explore further
and cross more dipole fields.

The CPU4 is thus \emph{doubly robust} to magnetic anomalies: same-frame
cancellation neutralises the constant (spatially uniform) component
(direction B result), and the spatially varying residual is small
within typical foraging radii (direction \(A \!\times\! B\) result). The
binding constraints remain \(\lambda T < 1\) and \(C \gtrsim 0.1\).

\begin{figure}[htbp]
\centering
\includegraphics[width=0.85\textwidth]{fig_axb_cancellation.png}
\caption{\label{fig:org3fe58eb}Constant compass bias (left, flat at 2.4 BL) versus spatially varying anomaly (right, grows with strength). The anomaly effect is small (\(\sim 2\times\) at \(12\,\mu\mathrm{T}\)) and substantially weaker than the constant-bias null result would suggest.}
\end{figure}
\subsection{Compass model comparison for path integration}
\label{sec:orgcda32ef}
How many cryptochrome molecules does a path integrator need?
Figure \ref{fig:orge0f567a} shows homing error versus \(\Ncry\) for three
representative compass models:

\begin{figure}[htbp]
\centering
\includegraphics[width=0.85\textwidth]{fig_pi_compass.png}
\caption{\label{fig:orge0f567a}Compass model comparison for path integration. Left: homing error versus \(\Ncry\). \(\fadox\) (\(C = 0.15\)) converges at \(\Ncry \sim 10\); \(\fadtrp\) (\(C = 0.01\)) requires \(\Ncry \gtrsim 200\) to match. Right: exploration duration sweep.}
\end{figure}

\(\fadox\) (\(C = 0.15\)) converges rapidly: \(\Ncry = 5\)
already gives 2.7 BL error, and \(\Ncry = 50\) gives 2.4 BL.
Relaxed \(\fadox\) (\(C = 0.10\)) requires \(\Ncry \sim 50\)
for comparable performance. \(\fadtrp\) (\(C = 0.01\)) is
dramatically worse: at \(\Ncry = 5\), homing error is 111 BL
(effectively lost); recovery requires \(\Ncry \gtrsim 200\) to bring the
error below 5 BL.

This produces a testable prediction: if the radical pair is
\(\fadtrp\), insects performing path integration must
have \(\gtrsim 200\) cryptochromes contributing to the compass signal per
integration step, or possess an alternative noise-reduction mechanism
not captured by our model.
\section{Discussion}
\label{sec:orge76c2f3}
\subsection{The contrast threshold as gatekeeper}
\label{sec:org3fc28a7}

The sharpness of the \(C \sim 0.1\) navigation threshold
(section \ref{sec:orgecec3d3}) has implications for the experimental
controversy. The Bassetto et al. \cite{bassetto2023} negative result
in \emph{Drosophila} and the Bradlaugh et al. \cite{bradlaugh2023}
positive result may not be contradictory if the compass contrast in
their respective preparations falls on opposite sides of this
threshold. Small differences in cryptochrome expression level,
cofactor availability, or radical-pair lifetime could move the system
across the phase boundary. The threshold is not a gradual degradation
--- it is a cliff.

This also explains why the choice between \(\fadox\) and
\(\fadtrp\) matters less than commonly assumed. Both
pairs work identically \emph{above} the threshold. The question is not
which produces a larger signal, but which has sufficient margin against
the cumulative suppression (relaxation, disorder, rate asymmetry) to
remain above \(C \sim 0.1\) under physiological conditions.
\(\fadox\) has \(\sim 3\times\) margin;
\(\fadtrp\) has \(\sim 1.5\times\), making it vulnerable
to any additional suppression mechanism.
\subsection{Same-frame cancellation: a general result}
\label{sec:orgf488eab}

The exact cancellation of constant compass bias in path integration
(Eq. \ref{eq:same-frame}) is, to our knowledge, a novel theoretical
result. It follows from a symmetry of the CPU4 architecture: both
integration and readout occur in the compass frame. Any rotation of
this frame (by a constant bias) is invisible to the homing
computation. This is \emph{not} an approximation --- it is an exact identity
that holds for arbitrary path geometry and bias magnitude.

The result has implications beyond radical-pair compasses. \emph{Any}
compass-based path integrator using a population-vector code with
matched integration and readout frames will exhibit the same
cancellation. This includes arthropod path integration using
polarised-light compasses or visual landmarks, provided the
``compass'' has a consistent rotational offset from the true heading.
The practical consequence: compass calibration errors do not
accumulate in path integration, making PI systems far more robust than
dead-reckoning analyses would suggest.

The cancellation breaks down when the bias varies spatially (our
direction \(A \!\times\! B\)), temporally (a drifting calibration), or
between the integration and readout phases (if the compass uses
different sensor pools for accumulation and homing). The spatial case
is the physically relevant one for geological anomalies, and our
results show it is second-order within typical foraging radii.
\subsection{Predictions}
\label{sec:org8fc6f5c}

The model generates three testable predictions:

\begin{enumerate}
\item \textbf{Cryptochrome number for path integration.} If \(\fadtrp\) is the operative radical pair, insects performing
path-integration tasks (homing from foraging) require \(\gtrsim 200\)
cryptochrome molecules contributing to the compass signal per
neural integration step. This could be tested by quantifying Cry4a
expression in the retinae of path-integrating versus
non-path-integrating species.

\item \textbf{RF disruption of path integration.} Radiofrequency fields at the
electron Larmor frequency (\(\sim 1.4\,\mathrm{MHz}\)) that
disrupt the radical-pair compass \cite{ritz2004,ritz2009} should
impair path integration in proportion to the increase in compass
noise, \emph{not} by introducing a systematic bias. PI should be robust
to a constant-offset RF field but sensitive to a spatially
inhomogeneous one.

\item \textbf{Memory leak detection.} The \(\lambda T = 1\) phase boundary
(section \ref{sec:org7b8c782}) predicts a sharp transition in homing
accuracy as foraging duration increases. This could be tested by
forcing insects to forage for controlled durations before allowing
homing, predicting a critical duration beyond which homing
performance collapses.
\end{enumerate}
\subsection{Limitations}
\label{sec:orgfb11200}

The model makes several simplifying assumptions:

\begin{itemize}
\item \textbf{Two-dimensional.} The axial ambiguity of the radical-pair compass
(\(\cos 2\alpha\) symmetry) is resolved by assumption. A full 3D
model would need to incorporate inclination sensing and gravity
cues for disambiguation.
\item \textbf{No landmark cues.} The ring attractor receives only magnetic
input. Real insects use multi-sensory integration (visual landmarks,
polarised light, wind direction) to stabilise the compass.
\item \textbf{Simplified spin model.} Our largest Hilbert space is \(16 \times 16\)
(two nuclei). Full FAD models require dim \(\sim 8000\) and may
reveal interactions not captured by our toy models.
\item \textbf{No axial ambiguity resolution.} We assume the \([0, 2\pi)\) compass;
real insects must resolve the \$\(\pi\)\$-ambiguity, which adds noise.
\item \textbf{Single integration frame.} The CPU4 circuit uses one compass for
both integration and readout. A system that integrates using one
sensor modality and reads out using another would not benefit from
same-frame cancellation.
\end{itemize}
\section{Methods}
\label{sec:org24d62d5}
\subsection{Spin chemistry computation}
\label{sec:org1eb1e69}

The spin Hamiltonian (Eq. \ref{eq:hamiltonian}) is constructed in the
uncoupled basis \(|m_{S_A}, m_{S_B}, m_{I_1}, \ldots\rangle\) for
Hilbert-space dimensions 8 (toy: 2 electrons + 1 nucleus), 16
(intermediate: 2 electrons + 2 nuclei), and 64 (intermediate + TrpH
\$\(\beta\)\$-proton). Hyperfine tensors: FAD N5 with
\(\mathbf{A} = \mathrm{diag}(0, 0, 1.046)\,\mathrm{mT}\)
(\(a_{\mathrm{iso}} = 523\,\mu\mathrm{T}\)); FAD N10 with
\(\mathbf{A} = \mathrm{diag}(0, 0, 0.378)\,\mathrm{mT}\); TrpH
\$\(\beta\)\$-protons at 0.71 and 1.07 mT. For \(\fadox\),
radical B carries no nuclear spins. Exchange coupling \(J = 0\) and
dipolar coupling set to zero for the models presented (their effects
are subsumed into the \(T_2\) relaxation parameter).

The master equation (Eq. \ref{eq:master}) is integrated using
\texttt{scipy.linalg.expm} for the coherent part and analytic
Lindblad dissipators with \(T_1 = T_2\). Recombination rates:
\(k_S = k_T = 10^6\,\mathrm{s}^{-1}\) (equal-rate default). The
singlet yield is computed by numeric quadrature over
\(t \in [0, 10/k_S]\) at 1000 time points, sweeping
\(\theta \in [0^{\circ}, 180^{\circ}]\) in \(2^{\circ}\) steps.
\subsection{Navigation simulation}
\label{sec:org5fe67c6}

The fast simulation uses the Ornstein--Uhlenbeck heading process
(Eq. \ref{eq:heading-update}) with Euler--Maruyama integration at
\(dt = 0.1\,\mathrm{s}\). The compass noise is derived analytically
from the Fisher information of the \(\cos 2\alpha\) signal
(Eq. \ref{eq:compass-noise}). Ensembles of 200 bugs per parameter
point, simulated for 2000 body-lengths travel distance on a
\(1000 \times 1000\) BL arena. Heading error is computed as the
circular mean absolute deviation from \(\theta_{\mathrm{goal}}\) over
the last 20\% of the trajectory (steady state). The fast simulation is
validated against the full ring-attractor simulation at \(C = 0.15\)
(agreement within 1--5\textsuperscript{\^{}}) and diverges at \(C = 0.01\) where the ring
attractor locks to noise.
\subsection{Anomaly field generation}
\label{sec:orgdf4b846}

Magnetic anomalies are modelled as buried vertical magnetic dipoles
with field:
\(\delta B_h(\mathbf{r}) \propto m \cdot d / (d^2 + \rho^2)^{5/2}\)
where \(m\) is the magnetic moment, \(d\) the burial depth, and \(\rho\)
the horizontal distance. The deviation angle
\(\delta\varphi = \arctan(\delta B_h / B_h)\) is pre-computed on a
\(150 \times 150\) grid covering the arena and bilinearly interpolated
at each timestep to avoid per-step evaluation.

Dipole positions and moments are drawn uniformly; burial depth
defaults to \(d = 80\) BL. Fault anomalies use a \(\tanh\) transition
profile; gradients are linear. The combined anomaly field is the
superposition of all sources.
\subsection{Path integration protocol}
\label{sec:org3e9bb43}

Outbound phase: the bug performs a correlated random walk
(\(\theta_{\mathrm{goal}}\) drawn from a wrapped Cauchy distribution
resampled every \(50\,\mathrm{s}\)) for a controlled exploration
duration \(T_{\mathrm{explore}}\). During exploration, the CPU4 array
(Eq. \ref{eq:cpu4}) continuously accumulates. At \(t =
T_{\mathrm{explore}}\), the homing phase begins: the steering law
switches to \(\theta_{\mathrm{goal}} = \theta_{\mathrm{home}}\) from
the CPU4 readout (Eq. \ref{eq:home-vector}), updated at each
timestep. The homing phase runs until the bug is within 1 BL of the
origin or \(t = T_{\mathrm{explore}} + 2000\,\mathrm{s}\). Homing
error is the final distance from the origin.
\subsection{Compass model comparison}
\label{sec:org3e65add}

Seven compass models are compared in matched ensembles:

\begin{table}[htbp]
\caption{\label{tab:org2dbff42}Summary of the seven compass models.}
\centering
\begin{tabular}{lllrrr}
\toprule
Model & Pair & Nuclei & Dim & \(C_0\) & \(C\) (\(T_2 = 3\,\mu\mathrm{s}\))\\
\midrule
toy-O2 & FAD-O\textsubscript{2} & N5 & 8 & 0.391 & 0.275\\
toy-TrpH & FAD-TrpH & N5 & 16 & 0.210 & 0.155\\
inter-O2 & FAD-O\textsubscript{2} & N5+N10 & 16 & 0.448 & 0.304\\
inter-TrpH & FAD-TrpH & N5+N10 & 64 & 0.245 & 0.158\\
analytical (\(C=0.15\)) & --- & --- & --- & 0.15 & 0.15\\
relaxed-O2 & FAD-O\textsubscript{2} & N5 & 8 & --- & 0.10\\
low-contrast & --- & --- & --- & 0.01 & 0.01\\
\bottomrule
\end{tabular}
\end{table}
\section{References}
\label{sec:orgd87f10c}
\begin{thebibliography}{30}

\bibitem[Schulten et~al.(1978)]{schulten1978}
Schulten K, Swenberg CE, Weller A (1978)
A biomagnetic sensory mechanism based on magnetic field modulated coherent electron spin motion.
\textit{Z. Phys. Chem. NF} \textbf{111}:1--5.

\bibitem[Ritz et~al.(2000)]{ritz2000}
Ritz T, Adem S, Schulten K (2000)
A model for photoreceptor-based magnetoreception in birds.
\textit{Biophys. J.} \textbf{78}:707--718.

\bibitem[Hore \& Mouritsen(2016)]{hore2016}
Hore PJ, Mouritsen H (2016)
The radical-pair mechanism of magnetoreception.
\textit{Annu. Rev. Biophys.} \textbf{45}:299--344.

\bibitem[Lee et~al.(2014)]{lee2014}
Lee AA, Lau JCS, Hogben HJ, Biskup T, Kattnig DR, Hore PJ (2014)
Alternative radical pairs for cryptochrome-based magnetoreception.
\textit{J. R. Soc. Interface} \textbf{11}:20131063.

\bibitem[Hiscock et~al.(2016)]{hiscock2016}
Hiscock HG, Worster S, Kattnig DR, Steber C, Mouritsen H, Hore PJ (2016)
The quantum needle of the avian magnetic compass.
\textit{PNAS} \textbf{113}:4634--4639.

\bibitem[Worster et~al.(2017)]{worster2017}
Worster S, Kattnig DR, Hore PJ (2017)
Spin relaxation of radicals in cryptochrome and its role in avian magnetoreception.
\textit{Sci. Rep.} \textbf{7}:11640.

\bibitem[Wong et~al.(2023)]{wong2023}
Wong SY, Frederiksen A, Hanber M, et~al.\ (2023)
Navigation of a magnetic field by a radical pair mechanism.
\textit{PCCP} \textbf{25}:1811.

\bibitem[Player \& Hore(2024)]{player2024}
Player TC, Hore PJ (2024)
Viability of superoxide-containing radical pairs for magnetoreception.
\textit{Nat. Commun.} \textbf{15}:10823.

\bibitem[Xu et~al.(2021)]{xu2021}
Xu J, Jarocha LE, Zollitsch T, et~al.\ (2021)
Magnetic sensitivity of cryptochrome 4 from a migratory songbird.
\textit{Nature} \textbf{594}:535--540.

\bibitem[Bradlaugh et~al.(2023)]{bradlaugh2023}
Bradlaugh AA, Fedele G, Sheridan AL, Sherwin AC, Sherwin AM,
Sherwin JC, et~al.\ (2023)
Essential elements of radical pair magnetosensing in \textit{Drosophila}.
\textit{Nature} \textbf{615}:111--116.

\bibitem[Bassetto et~al.(2023)]{bassetto2023}
Bassetto M, Reichl T, Kobylkov D, et~al.\ (2023)
No evidence for magnetic field effects on the behaviour of \textit{Drosophila}.
\textit{Nature} \textbf{620}:595--599.

\bibitem[Ritz et~al.(2004)]{ritz2004}
Ritz T, Thalau P, Phillips JB, Wiltschko R, Wiltschko W (2004)
Resonance effects indicate a radical-pair mechanism for avian magnetic compass.
\textit{Nature} \textbf{429}:177--180.

\bibitem[Ritz et~al.(2009)]{ritz2009}
Ritz T, Wiltschko R, Hore PJ, Rodgers CT, Stapput K, Thalau P, Timmel CR, Wiltschko W (2009)
Magnetic compass of birds is based on a molecule with optimal directional sensitivity.
\textit{Biophys. J.} \textbf{97}:3154--3160.

\bibitem[Engels et~al.(2014)]{engels2014}
Engels S, Schneider NL, Lefeldt N, et~al.\ (2014)
Anthropogenic electromagnetic noise disrupts magnetic compass orientation in a migratory bird.
\textit{Nature} \textbf{509}:353--356.

\bibitem[Nordmann et~al.(2017)]{nordmann2017}
Nordmann GC, Hochstoeger T, Keays DA (2017)
Magnetoreception --- a sense without a receptor.
\textit{PLOS Biology} \textbf{15}:e2003234.

\bibitem[Mouritsen(2018)]{mouritsen2018}
Mouritsen H (2018)
Long-distance navigation and magnetoreception in migratory animals.
\textit{Nature} \textbf{558}:50--59.

\bibitem[Mouritsen et~al.(2005)]{mouritsen2005}
Mouritsen H, Janssen-Bienhold U, Liedvogel M, et~al.\ (2005)
Cryptochromes and neuronal-activity markers colocalize in the retina of migratory birds during magnetic orientation.
\textit{PNAS} \textbf{102}:8339--8344.

\bibitem[Zapka et~al.(2009)]{zapka2009}
Zapka M, Heyers D, Hein CM, et~al.\ (2009)
Visual but not trigeminal mediation of magnetic compass information in a migratory bird.
\textit{Nature} \textbf{461}:1274--1278.

\bibitem[Dreyer et~al.(2018)]{dreyer2018}
Dreyer D, Frost B, Mouritsen H, Gunther A, Green K, Whitehouse M, Johnsen S, Heinze S, Warrant E (2018)
The Earth's magnetic field and visual landmarks steer migratory flight behavior in the nocturnal Australian bogong moth.
\textit{Curr. Biol.} \textbf{28}:2160--2166.

\bibitem[Kim et~al.(2017)]{kim2017}
Kim SS, Rouault H, Druckmann S, Jayaraman V (2017)
Ring attractor dynamics in the \textit{Drosophila} central brain.
\textit{Science} \textbf{356}:849--853.

\bibitem[Seelig \& Jayaraman(2015)]{seelig2015}
Seelig JD, Jayaraman V (2015)
Neural dynamics for landmark orientation and angular path integration.
\textit{Nature} \textbf{521}:186--191.

\bibitem[Stone et~al.(2017)]{stone2017}
Stone T, Webb B, Adden A, Weddig NB, Honkanen A, Templin R, Wcislo W, Scimeca L, Warrant E, Heinze S (2017)
An anatomically constrained model for path integration in the bee brain.
\textit{Curr. Biol.} \textbf{27}:3069--3085.

\bibitem[Kim \& Turner-Evans(2024)]{kim2024}
Kim SS, Turner-Evans DB (2024)
Ring attractor dynamics emerge from a spiking model of the entire \textit{Drosophila} protocerebral bridge.
\textit{bioRxiv} doi:10.1101/2024.

\bibitem[Hulse et~al.(2021)]{hulse2021}
Hulse BK, Haberkern H, Franconville R, et~al.\ (2021)
A connectome of the \textit{Drosophila} central complex reveals network motifs suitable for flexible navigation and context-dependent action selection.
\textit{eLife} \textbf{10}:e66039.

\bibitem[Goulard et~al.(2023)]{goulard2023}
Goulard R, Heinze S, Webb B (2023)
An integrative computational model of the insect central complex for generating heading and steering signals.
\textit{PLoS Comput. Biol.} \textbf{19}:e1011480.

\bibitem[Heinze \& Reppert(2011)]{heinze2011}
Heinze S, Reppert SM (2011)
Sun compass integration of skylight cues in migratory monarch butterflies.
\textit{Neuron} \textbf{69}:345--358.

\bibitem[Kattnig et~al.(2016)]{kattnig2016}
Kattnig DR, Solov'yov IA, Hore PJ (2016)
Electron spin relaxation in cryptochrome-based magnetoreception.
\textit{Phys. Chem. Chem. Phys.} \textbf{18}:12443--12456.

\bibitem[Hochstoeger et~al.(2020)]{hochstoeger2020}
Hochstoeger T, Al Said T, Maaser D, et~al.\ (2020)
The biophysical, molecular, and anatomical landscape of pigeon CRY4.
\textit{Sci. Adv.} \textbf{6}:eabb9110.

\end{thebibliography}
\section{Supplementary Information}
\label{sec:orgaa8f748}
\subsection{Additional figures}
\label{sec:org66a7590}

The complete analysis generates 36 figures spanning all parameter
sweeps. The seven selected for the main text are representative; the
full set is available in the supplementary archive:

\begin{itemize}
\item Phase 1 validation: \texttt{fig\_trajectory.png}, \texttt{fig\_ensemble.png},
\texttt{fig\_validate.png}, \texttt{fig\_quantum\_trajectory.png},
\texttt{fig\_compare\_models.png}, \texttt{fig\_phase\_diagram.png}
\item Peclet analysis: \texttt{fig\_peclet.png}, \texttt{fig\_differentiate.png},
\texttt{fig\_critical\_noise.png}
\item Sensor population: \texttt{fig\_ncry\_sweep.png}, \texttt{fig\_ncry\_noise.png}
\item Validation: \texttt{fig\_validate\_fast.png}
\item Suppression mechanisms: \texttt{fig\_relax\_navigation.png},
\texttt{fig\_relax\_contrasts.png}, \texttt{fig\_relax\_margin.png},
\texttt{fig\_uneq\_rates.png}, \texttt{fig\_uneq\_nav.png}, \texttt{fig\_orient\_disorder.png},
\texttt{fig\_orient\_nav.png}
\item Anomalies: \texttt{fig\_anomaly\_maps.png}, \texttt{fig\_anomaly\_traj.png},
\texttt{fig\_anomaly\_sweep.png}, \texttt{fig\_anomaly\_critical.png}
\item Path integration: \texttt{fig\_pi\_homing.png}, \texttt{fig\_pi\_bias\_leak.png},
\texttt{fig\_pi\_phase.png}, \texttt{fig\_pi\_compass.png}, \texttt{fig\_pi\_noise.png},
\texttt{fig\_pi\_bias.png}, \texttt{fig\_pi\_anomaly.png}
\item Synthesis: \texttt{fig\_axb\_strength.png}, \texttt{fig\_axb\_cancellation.png},
\texttt{fig\_axb\_phase.png}, \texttt{fig\_axb\_compass.png}
\end{itemize}
\subsection{Parameter tables}
\label{sec:orgb9d78af}

\begin{table}[htbp]
\caption{\label{tab:org11ce8cb}Default simulation parameters.}
\centering
\begin{tabular}{llrll}
\toprule
Parameter & Symbol & Default & Unit & Notes\\
\midrule
Ring neurons & \(N\) & 8 & --- & Biological: 8--16\\
Ring time constant & \(\tau_r\) & 50 & ms & Scaled for sim speed\\
Excitatory weight & \(w_{\mathrm{exc}}\) & 1.5 & --- & Tuned for single bump\\
Inhibitory weight & \(w_{\mathrm{inh}}\) & 0.3 & --- & Global (\(\Delta 7\))\\
Magnetic gain & \(g_{\mathrm{mag}}\) & 0.5 & --- & \\
Compass contrast & \(C\) & 0.15 & --- & 15\% for FAD-O\(_2\)\\
Sensor noise & \(\sigma_{\mathrm{sensor}}\) & 0.05 & --- & Per molecule\\
Cryptochrome count & \(\Ncry\) & 50 & --- & Population averaging\\
Steering gain & \(\kappa\) & 1.0 & s\(^{-1}\) & \\
Speed & \(v\) & 1.0 & BL/s & \\
Angular noise & \(\sigma_\theta\) & 0.1 & rad/\(\sqrt{\mathrm{s}}\) & Motor + compass\\
Translational noise & \(\sigma_x, \sigma_y\) & 0.05 & BL/\(\sqrt{\mathrm{s}}\) & Wind/terrain\\
Goal heading & \(\theta_{\mathrm{goal}}\) & \(3\pi/4\) & rad & SW\\
Time step & \(dt\) & 0.1 & s & Fast simulation\\
Recombination rate & \(k_S = k_T\) & \(10^6\) & s\(^{-1}\) & Equal-rate default\\
Ensemble size & --- & 200 & bugs & Per parameter point\\
Arena size & --- & 1000 & BL & Square domain\\
CPU4 neurons & --- & 8 & --- & Preferred dirs \(2\pi i/8\)\\
CPU4 leak rate & \(\lambda\) & 0 & s\(^{-1}\) & Perfect integrator default\\
Anomaly grid & --- & \(150 \times 150\) & --- & Bilinear interpolation\\
Dipole burial depth & \(d\) & 80 & BL & \\
\bottomrule
\end{tabular}
\end{table}
\subsection{Equation summary}
\label{sec:org80819b6}

For reference, the twelve key equations of the model:

\begin{center}
\begin{tabular}{lll}
Eq. & Name & Section\\
\hline
\ref{eq:hamiltonian} & Spin Hamiltonian & \ref{sec:org57901cc}\\
\ref{eq:master} & Haberkorn master equation & \ref{sec:org57901cc}\\
\ref{eq:singlet-yield} & Singlet yield & \ref{sec:org57901cc}\\
\ref{eq:contrast} & Compass contrast & \ref{sec:org57901cc}\\
\ref{eq:ring-attractor} & Ring attractor dynamics & \ref{sec:org4256fba}\\
\ref{eq:compass-noise} & Compass noise & \ref{sec:org4256fba}\\
\ref{eq:heading-update} & Heading update (steering) & \ref{sec:org16ac780}\\
\ref{eq:peclet} & Peclet number & \ref{sec:org16ac780}\\
\ref{eq:cpu4} & CPU4 update & \ref{sec:org1862aca}\\
\ref{eq:home-vector} & Home vector decode & \ref{sec:org1862aca}\\
\ref{eq:same-frame} & Same-frame cancellation & \ref{sec:org048a245}\\
\ref{eq:critical-leak} & Critical leak rate & \ref{sec:org7b8c782}\\
\end{tabular}
\end{center}
\subsection{Computational note}
\label{sec:org8e15fed}

This work was conducted as a human--AI collaboration using MāyāLucIA,
a literate-programming research environment. The human (mu2tau)
provided the scientific direction, domain expertise, and critical
evaluation; Claude (Anthropic) assisted with model implementation,
parameter sweeps, and manuscript preparation. All computational code
(Python) and this document (Org-mode) are available at the project
repository.
\end{document}
